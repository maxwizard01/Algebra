\documentclass{article}
\usepackage[utf8]{inputenc}
\usepackage{blindtext}
\usepackage[a4paper, total={7in, 10in},left=10mm,]{geometry}
\usepackage{tikz}
\usepackage{bbding}
\usepackage{pifont}
\usepackage{wasysym}
\usepackage{amssymb}
\usepackage{amsmath,amssymb}
\usepackage{mathptmx}
\usetikzlibrary{shapes.geometric}
\usepackage[nomessages]{fp}% http://ctan.org/pkg/fp
\definecolor{B}{HTML}{2E79B2}
\definecolor{W}{HTML}{FF0000}
\usepackage{enumerate}
\usepackage{parcolumns}
%\setcounter{section}{0}

\title{GROUP 1}
\author{Project 1}

\begin{document}

\section{PROJECT 1}
\subsection{Tutorial I}
\large{}
1. Let $\xi \in \mathbb{R}^{n}$ and $\|\cdot\|_{0},\|\cdot\|_{1}$ and $\|\cdot\|_{u}$ be norms on $\mathbb{R}^{n}$ defined by

$$
\|\xi\|_{0}=\left(\sum_{i=1}^{n} x_{i}^{2}\right)^{\frac{1}{2}}, \quad\|\xi\|_{1}=\sum_{i=1}^{n}\left|x_{i}\right|, \quad\|\xi\|_{u}=\max _{1 \leq i \leq n}\left|x_{i}\right|
$$

where $\xi=\left(x_{1}, x_{2}, \ldots, x_{n}\right)$, with $x_{i} \in \mathbb{R}, i=1,2, \ldots, n$.
\begin{itemize}
    \item Prove that
$$
\frac{1}{\sqrt{n}}\|\xi\|_{0} \leq\|\xi\|_{u} \leq\|\xi\|_{1}, \quad \forall_{\xi} \in \mathbb{R}^{n}
$$
\item Hence, deduce that the function

$$
f:\left(\mathbb{R}^{n},\|\cdot\|_{1}\right) \longrightarrow\left(\mathbb{R}^{n},\|\cdot\|_{0}\right)~~~~~~~~is ~continuous~ on ~~\mathbb{R}^{n}
$$
\end{itemize}

\textbf{Solution}\\

let $x_k$ be any entries in $\xi$ such that $1 \leq k \leq n$, it is obvious that

$$
x_{k}^{2} \leq\left(\left|x_{1}\right|+\left|x_{2}\right|+\cdots+\left|x_{n}\right|\right)^{2} .
$$
which can be rewrite in summation notation as :

$$\begin{equation}
x_{k}^{2} \leq \left(\sum_{i=1}^{n}\left|x_{i}\right|\right)^{2}
\end{equation}$$

Now Summing both sides of the equation(1) from $k=1$ to $n$, we have

$$\begin{equation}
\sum_{k=1}^{n} x_{k}^{2} \leq \sum_{k=1}^{n}\left(\sum_{i=1}^{n}\left|x_{i}\right|\right)^{2}
\end{equation}$$

Since $k$ is a dummy variable on the left side of equation(2), we can change it to $i$ and have

$$
\sum_{i=1}^{n} x_{i}^{2} \leq \sum_{k=1}^{n}\left(\sum_{i=1}^{n}\left|x_{i}\right|\right)^{2}
$$

Shifting the first summation to the start we have
$$
\begin{gathered}
\sum_{i=1}^{n} x_{i}^{2} \leq\left(\sum_{i=1}^{n}\left|x_{i}\right|\right)^{2} \sum_{k=1}^{n} 1 \implies
\sum_{i=1}^{n} x_{i}^{2} \leq n\left(\sum_{i=1}^{n}\left|x_{i}\right|\right)^{2}  
\end{gathered}~ ~~\left(since ~\sum_{k=1}^{n} 1=n\right)
$$

Now taking the square root of both side, we have
$$
\begin{aligned}
& \left(\sum_{i=1}^{n} x_{i}^{2}\right)^{\frac{1}{2}} \leq \sqrt{n} \sum_{i=1}^{n}\left|x_{i}\right|
\end{aligned}
$$

Divide both side by $\sqrt{n}$ we have
$$ \frac{1}{\sqrt{n}}\left(\sum_{i=1}^{n} x_{i}^{2}\right)^{\frac{1}{2}} \leq \sum_{i=1}^{n}\left|x_{i}\right|
$$

Recall that, $$\|\xi\|_{0}=\left(\sum_{i=1}^{n} x_{i}^{2}\right)^{\frac{1}{2}} ~and ~\quad\|\xi\|_{1}=\sum_{i=1}^{n}\left|x_{i}\right|$$

Therefore,

$$\frac{1}{\sqrt{n}}\left(\sum_{i=1}^{n} x_{i}^{2}\right)^{\frac{1}{2}} \leq \sum_{i=1}^{n}\left|x_{i}\right| .
\implies
\frac{1}{\sqrt{n}}\|\xi\|_{0} \leq\|\xi\|_{1}
$$\\

Firstly let us  show that

$$
\|\xi\|_{u} \leq\|\xi\|_{1}
$$

It is Obvious that

$$
\left|x_{1}\right|+\left|x_{2}\right|+\cdots+\left|x_{n}\right| \geq \max _{1 \leq i \leq n}\left|x_{i}\right| .
$$

the above equation can be rewrite in summation form as:

$$
\sum_{i=1}^{n}\left|x_{i}\right| \geq \max _{1 \leq i \leq n}\left|x_{i}\right|
$$

this shows that
$$
\|\xi\|_{u} \leq\|\xi\|_{1} \text {. }
$$

Now we need to show that.
$$
\frac{1}{\sqrt{n}}\|\xi\|_{0} \leq\|\xi\|_{u}
$$
we know that

$$
x_{i}^{2} \leq \max _{1 \leq i \leq n}\left|x_{i}\right|^{2}
$$

$$ \text{But} \max _{1 \leq i \leq n}\left|x_{i}\right|^{2}=\left(\max _{1 \leq i \leq n}\left|x_{i}\right|\right)^{2}$$
So, $$
x_{i}^{2} \leq \max _{1 \leq i \leq n}\left|x_{i}\right|^{2} \iff x_{i}^{2} \leq\left(\max _{1 \leq i \leq n}\left|x_{i}\right|\right)^{2}
$$

Take summation of both sides of from $i=1$ to $n$, we have

$$
\sum_{i=1}^{n} x_{i}^{2} \leq \sum_{i=1}^{n}\left(\max _{1 \leq i \leq n}\left|x_{i}\right|\right)^{2}
$$

Since the maximum function on the right side  is not depending on $i$, then we can shift it.

$$
\sum_{i=1}^{n} x_{i}^{2} \leq\left(\max _{1 \leq i \leq n}\left|x_{i}\right|\right)^{2} \sum_{i=1}^{n} 1
$$
$$\text{Since } ~~\sum_{i=1}^{n} 1 =n ~~\text{ then we have,} $$ 

$$
\sum_{i=1}^{n} x_{i}^{2} \leq n\left(\max _{1 \leq i \leq n}\left|x_{i}\right|\right)^{2}
$$

Since both sides are positive, we can take square root and obtain the following:

$$
\left(\sum_{i=1}^{n} x_{i}^{2}\right)^{\frac{1}{2}} \leq \sqrt{n} \max _{1 \leq i \leq n}\left|x_{i}\right|, $$
divide both side by $\sqrt{n}$ we have:
$$\quad \frac{1}{\sqrt{n}}\left(\sum_{i=1}^{n} x_{i}^{2}\right)^{\frac{1}{2}} \leq \max _{1 \leq i \leq n}\left|x_{i}\right|
$$

Showing that

$$
\frac{1}{\sqrt{n}}\|\xi\|_{0} \leq\|\xi\|_{u}
$$

From the workings above we showed that the following are true:

$$
\|\xi\|_{u} \leq\|\xi\|_{1}, \quad \frac{1}{\sqrt{n}}\|\xi\|_{0} \leq\|\xi\|_{u}, \quad \frac{1}{\sqrt{n}}\|\xi\|_{0} \leq\|\xi\|_{1}
$$

Then we can deduce from those three fact that:
$$\frac{1}{\sqrt{n}}\|\xi\|_{0} \leq\|\xi\|_{u} \leq\|\xi\|_{1}, \quad \forall \xi \in \mathbb{R}^{n}$$
And this completed the proof.



\textbf{For Second Part now}\\

Before we do the prove the following Lemma is needed.

\textbf{Lemma :}  Let $\xi, \eta \in \mathbb{R}^{n}$, where $\xi=\left(x_{1}, x_{2}, \ldots, x_{n}\right)$ and $\eta=\left(y_{1}, y_{2}, \ldots, y_{n}\right)$, with $x_{i}, y_{i} \in \mathbb{R}$, for $i=1$ to $n$. Then

$$
||\left|\xi\left\|_{0}-\right\| \eta\left\|_{0} \mid \leq\right\| \xi-\eta \|_{0}, \quad \xi, \eta \in \mathbb{R}^{n}\right.
$$\\
Proof:\\

Consider $\mathbb{R}^{n}$ and let $\xi, \eta \in \mathbb{R}^{n}$. Define the function

$$
\rho: \mathbb{R}^{n} \times \mathbb{R}^{n} \longrightarrow \mathbb{R}^{+}
$$
by
$$
\rho(\xi, \eta)=\left(\sum_{i=1}^{n}\left(x_{i}-y_{i}\right)^{2}\right)^{\frac{1}{2}}
$$

Let $\zeta=\left(z_{1}, z_{2}, \ldots, z_{n}\right) \in \mathbb{R}^{n}$. Then

$$
\begin{aligned}
{[\rho(\xi, \zeta)]^{2} } & =\sum_{i=1}^{n}\left(x_{i}-z_{i}\right)^{2}=\sum_{i=1}^{n}\left(x_{i}-y_{i}+y_{i}-z_{i}\right)^{2} \\
& =\sum_{i=1}^{n}\left[\left(x_{i}-y_{i}\right)^{2}+\left(y_{i}-z_{i}\right)^{2}+2\left|x_{i}-y_{i}\right|\left|y_{i}-z_{i}\right|\right] \\
& =\sum_{i=1}^{n}\left(x_{i}-y_{i}\right)^{2}+\sum_{i=1}^{n}\left(y_{i}-z_{i}\right)^{2}+2 \sum_{i=1}^{n}\left|x_{i}-y_{i}\right|\left|y_{i}-z_{i}\right|
\end{aligned}
$$
\newpage
But, According to Cauchy Inequity:
$$
\sum_{i=1}^{n}\left|x_{i}\right|\left|y_{i}\right| \leq\left(\sum_{i=1}^{n}x_{i}^{2}\right)^{\frac{1}{2}}\left(\sum_{i=1}^{n}y_{i}^{2}\right)^{\frac{1}{2}} .
$$
which means that the following is also true:.
$$
\sum_{i=1}^{n}\left|x_{i}-y_{i}\right|\left|y_{i}-z_{i}\right| \leq\left(\sum_{i=1}^{n}\left(x_{i}-y_{i}\right)^{2}\right)^{\frac{1}{2}}\left(\sum_{i=1}^{n}\left(y_{i}-z_{i}\right)^{2}\right)^{\frac{1}{2}} .
$$

SO,
$$
\begin{aligned}
{[\rho(\xi, \zeta)]^{2} } \leq \sum_{i=1}^{n}\left(x_{i}-y_{i}\right)^{2}+\sum_{i=1}^{n}\left(y_{i}-z_{i}\right)^{2}+2\left(\sum_{i=1}^{n}\left(x_{i}-y_{i}\right)^{2}\right)^{\frac{1}{2}}\left(\sum_{i=1}^{n}\left(y_{i}-z_{i}\right)^{2}\right)^{\frac{1}{2}}
\end{aligned}
$$

Therefore

$$
\begin{aligned}
{[\rho(\xi, \zeta)]^{2} } & \leq[\rho(\xi, \eta)]^{2}+[\rho(\eta, \zeta)]^{2}+2 \rho(\xi, \eta) \rho(\eta, \zeta) \\
& \leq[\rho(\xi, \eta)+\rho(\eta, \zeta)]^{2}
\end{aligned}
$$

Since bothsides are greater than 0, we can take the square root on both sides and obtain


\begin{equation}
\rho(\xi, \zeta) \leq \rho(\xi, \eta)+\rho(\eta, \zeta)
\end{equation}

Setting $\zeta=\overline{0}$ in inequality (3), we have

$$
\rho(\xi, \overline{0}) \leq \rho(\xi, \eta)+\rho(\eta, \overline{0})
$$

Showing that

$$
\left(\sum_{i=1}^{n} x_{i}^{2}\right)^{\frac{1}{2}} \leq\left(\sum_{i=1}^{n}\left(x_{i}-y_{i}\right)^{2}\right)^{\frac{1}{2}}+\left(\sum_{i=1}^{n} y_{i}^{2}\right)^{\frac{1}{2}}
$$

which can also be written as 

$$
\|\xi\|_{0} \leq\|\xi-\eta\|_{0}+\|\eta\|_{0}
$$
Take $\|\eta\|_{0}$ to Left side we have

$$
\|\xi\|_{0}-\|\eta\|_{0} \leq\|\xi-\eta\|_{0}
$$

Replacing $\xi$ with $\eta$ in (2.12), we obtain

$$
\|\eta\|_{0}-\|\xi\|_{0}=-\left(\|\xi\|_{0}-\|\eta\|_{0}\right) \leq\|\eta-\xi\|_{0}=\|-(\xi-\eta)\|_{0}=\|\xi-\eta\|_{0}
$$

Now, we have that

$$
-\left(\|\xi\|_{0}-\|\eta\|_{0}\right) \leq\|\xi-\eta\|_{0} \text { and }\|\xi\|_{0}-\|\eta\|_{0} \leq\|\xi-\eta\|_{0}
$$

Both inequalities imply

$$
-\left(\|\xi\|_{0}-\|\eta\|_{0}\right) \leq\|\xi\|_{0}-\|\eta\|_{0} \leq\|\xi-\eta\|_{0}
$$

Hence,

$$
\left| \|\xi\|_{0}-\|\eta\|_0 \left|\leq\right\| \xi-\eta \|_{0}\right.
$$



\large{\textbf{Now Let us prove for the continuous}}\\

(b) Let $\xi, \eta \in \mathbb{R}^{n}$, such that \\
$$\xi=\left(x_{1}, x_{2}, \ldots, x_{n}\right),
\text{ and } \eta=\left(y_{1}, y_{2}, \ldots, y_{n}\right), \text{ with } x_i,y_{i} \in \mathbb{R},\text{ for i=1,2, }\ldots, n$$. 

To show that the function $f$ defined by $f:\left(\mathbb{R}^{n},\|\cdot\|_{1}\right) \longrightarrow\left(\mathbb{R}^{n},\|\cdot\|_{0}\right)$ is continuous on $\mathbb{R}^{n}$,

given $\epsilon>0$, we have to find a $\delta>0$, such that 
$$\left|\|\xi\|_{1}-\|\eta\|_{1}\right|<\delta \implies \left|f\left(\|\xi\|_{1}\right)-f\left(\| \eta||_1\right)\right|= ||\left|\xi\left\|_{0}-\right\| \eta \|_{0}\right|<\epsilon$$.

Now,

$$
\left|f\left(\|\xi\|_{1}\right)-f\left(\left\|\eta_{1}\right\|\right)\right|=\left|\|\xi\|_{0}-\|\eta\|_{0}\right|
$$

Using the inequality  (Lemma 2)

$$
\left|\|\xi\|_{0}-\|\eta\|_{0}\right| \leq\|\xi-\eta\|_{0}, \quad \xi, \eta \in \mathbb{R}^{n}
$$

So, 

$$
\left|f\left(\|\xi\|_{1}\right)-f\left(\| \eta_{1}||\right)\right|=\left|\|\xi\|_{0}-\|\eta\|_{0}\right| \leq\|\xi-\eta\|_{0}
$$
Which can be deduce to;

$$
\left|f\left(\|\xi\|_{1}\right)-f\left(\left\|\eta_{1}\right\|\right)\right|\leq\|\xi-\eta\|_{0} .
$$

But recall that we Proved it earlier that :

$$\|\xi\|_{0} \leq \sqrt{n}\|\xi\|_{1} \implies  \|\xi-\eta\|_{0} \leq \sqrt{n}\|\xi-\eta\|_{1} $$

Applying that fact then we have;
$$
\left|f\left(\|\xi\|_{1}\right)-f\left(\left\|\eta_{1}\right\|\right)\right| \leq\|\xi-\eta\|_{0} \leq \sqrt{n}\|\xi-\eta\|_{1} .
$$

$$|\| \xi\left\|_{1}-\right\| \eta \|_{1} \mid<\delta$$, we derive

$$
\left|\|\xi\|_{1}-\|\eta\|_{1}\right| \leq\|\xi-\eta\|_{1}<\delta
$$

So, we have that

$$
\left\|\left|\xi\left\|_{1}-\right\| \eta\left\|_{1} \mid<\delta \Longrightarrow\right\| \xi-\eta \|_{1}<\delta\right.\right. \text {. }
$$

Choosing $\delta(\epsilon)=\frac{\epsilon}{\sqrt{n}}$, it follows that

$$
\left|f\left(\|\xi\|_{1}\right)-f\left(\left\|\eta_{1}\right\|\right)\right|<\epsilon .
$$

Showing that the function $f$ defined by $f:\left(\mathbb{R}^{n},\|\cdot\|_{1}\right) \longrightarrow\left(\mathbb{R}^{n},\|\cdot\|_{0}\right)$ is continuous on $\mathbb{R}^{n}$.
(c) If $\xi, \eta \in \mathbb{R}^{n}$,



\subsection{Tutorial II}
\large{}
If $\xi,\eta \in \mathbb{R}^n,$
prove that the Euclidean norm $\|\cdot\|_{0}$ satisfies the parallelogram identity

$$
\|\xi+\eta\|_{0}^{2}+\|\xi-\eta\|_{0}^{2}=2\left(\|\xi\|_{0}^{2}+\|\eta\|_{0}^{2}\right)
$$

Show also that

$$
\|\xi+\eta\|_{0}^{2}=\|\xi\|_{0}^{2}+\|\eta\|_{0}^{2}
$$

holds if and only if $\langle\xi, \eta\rangle=0$. In this case, we say that $\xi$ and $\eta$ are orthogonal.\\\\

\textbf{SOLUTION}\\

\textbf{Proof}.\\

Let $\xi, \eta \in \mathbb{R}^{n}$, such that $\xi=\left(x_{1}, x_{2}, \ldots, x_{n}\right)$ and $\eta=\left(y_{1}, y_{2}, \ldots, y_{n}\right)$ with $x_{i},y_i \in \mathbb{R}, i=1,2, \ldots, n$

$$
\|\xi+\eta\|_{0}^{2}=\left\|\left(x_{1}+y_{1}, x_{2}+y_{2}, \ldots, x_{n}+y_{n}\right)\right\|_{0}^{2}=\sum_{i=1}^{n}\left(x_{i}+y_{i}\right)^{2} .
$$

$$
\|\xi-\eta\|_{0}^{2}=\left\|\left(x_{1}-y_{1}, x_{2}-y_{2}, \ldots, x_{n}-y_{n}\right)\right\|_{0}^{2}=\sum_{i=1}^{n}\left(x_{i}-y_{i}\right)^{2}
$$
Combining both we can say,

$$
\begin{aligned}
\|\xi+\eta\|_{0}^{2}+\|\xi-\eta\|_{0}^{2} & =\sum_{i=1}^{n}\left(x_{i}+y_{i}\right)^{2}+\sum_{i=1}^{n}\left(x_{i}-y_{i}\right)^{2} \\
& =\sum_{i=1}^{n}\left(x_{i}^{2}+y_{i}^{2}+2 x_{i} y_{i}\right)+\sum_{i=1}^{n}\left(x_{i}^{2}+y_{i}^{2}-2 x_{i} y_{i}\right) \\
& =\sum_{i=1}^{n}\left(x_{i}^{2}+y_{i}^{2}+2 x_{i} y_{i}+x_{i}^{2}+y_{i}^{2}-2 x_{i} y_{i}\right) \\
& =\sum_{i=1}^{n}\left(2 x_{i}^{2}+2 y_{i}^{2}\right)=2 \sum_{i=1}^{n}\left(x_{i}^{2}+y_{i}^{2}\right) \\
& =2\left(\sum_{i=1}^{n} x_{i}^{2}+\sum_{i=1}^{n} y_{i}^{2}\right) \\
& =2\left(\|\xi\|_{0}^{2}+\|\eta\|_{0}^{2}\right) .
\end{aligned}
$$

Hence, the Euclidean norm $\|\cdot\|_{0}$ satisfies the parallelogram identity

$$
\|\xi+\eta\|_{0}^{2}+\|\xi-\eta\|_{0}^{2}=2\left(\|\xi\|_{0}^{2}+\|\eta\|_{0}^{2}\right)
$$
\newpage
\textbf{For second part now}\\
We want to show that  $$\langle\xi, \eta\rangle=0 \iff \|\xi+\eta\|_{0}^{2}=\|\xi\|_{0}^{2}+\|\eta\|_{0}^{2}$$.


Assume that $\langle\xi, \eta\rangle=0$

$$\begin{align}
\|\xi+\eta\|_{0}^{2}&=\sum_{i=1}^{n}\left(x_{i}+y_{i}\right)^{2}=\sum_{i=1}^{n}\left(x_{i}^{2}+y_{i}^{2}+2 x_{i} y_{i}\right)\\
&=\sum_{i=1}^{n} x_{i}^{2}+\sum_{i=1}^{n} y_{i}^{2}+2\sum_{i=1}^{n} x_iy_{i}\\
&=\|\xi\|_{0}^{2}+\|\eta\|_{0}^{2}
+2\langle \xi,\eta \rangle\\
&=\|\xi\|_{0}^{2}+\|\eta\|_{0}^{2} \text{ ( since }\langle \xi,\eta \rangle=0)
\end{align}$$


Hence, if $\langle\xi, \eta\rangle=0$, then $\|\xi+\eta\|_{0}^{2}=\|\xi\|_{0}^{2}+\|\eta\|_{0}^{2}$.\\

\textbf{Conversely, }\\

assume that $\|\xi+\eta\|_{0}^{2}=\|\xi\|_{0}^{2}+\|\eta\|_{0}^{2}$. Then,
$$
\begin{aligned}
\sum_{i=1}^{n}\left(x_{i}+y_{i}\right)^{2} & =\sum_{i=1}^{n} x_{i}^{2}+\sum_{i=1}^{n} y_{i}^{2} \\
\sum_{i=1}^{n}\left(x_{i}^{2}+y_{i}^{2}+2 x_{i} y_{i}\right) & =\sum_{i=1}^{n}\left(x_{i}^{2}+y_{i}^{2}\right) \\
\sum_{i=1}^{n}\left(x_{i}^{2}+y_{i}^{2}\right)+2 \sum_{i=1}^{n} x_{i} y_{i} & =\sum_{i=1}^{n}\left(x_{i}^{2}+y_{i}^{2}\right) \\
2 \sum_{i=1}^{n} x_{i} y_{i} & =\sum_{i=1}^{n}\left(x_{i}^{2}+y_{i}^{2}\right)-\sum_{i=1}^{n}\left(x_{i}^{2}+y_{i}^{2}\right)=0 .
\end{aligned}
$$

So,

$$
\sum_{i=1}^{n} x_{i} y_{i}=0
$$

Since $\langle\xi, \eta\rangle=\sum_{i=1}^{n} x_{i} y_{i}$, Then we can say $\langle\xi, \eta\rangle=0$.\\

Thus, if $\|\xi+\eta\|_{0}^{2}=\|\xi\|_{0}^{2}+\|\eta\|_{0}^{2}$, then $\langle\xi, \eta\rangle=0$.\\

In this case, $\xi$ and $\eta$ are orthogonal.



\newpage
\subsection{Tutorial III}
Prove the Young's inequality

$$
\sum_{i=1}^{n}\left|x_{i} y_{i}\right| \leq\left(\sum_{i=1}^{n}\left|x_{i}\right|^{p}\right)^{\frac{1}{p}}\left(\sum_{i=1}^{n}\left|y_{i}\right|^{q}\right)^{\frac{1}{q}}
$$

$$\text{where } \frac{1}{p}+\frac{1}{q}=1$$

\textbf{SOLUTION}\\

\large
Before we proceed to the proof the following lemma is needed\\

\textbf{Lemma }:\\
Let $f$ be a real-valued, continuous, and strictly increasing function on $[0, c]$ with $c>0$. If $f(0)=0, a \in[0, c]$, and $b \in[0, f(c)]$, then
\begin{equation}\tag{1.1}
\int_{0}^{a} f(x) \mathrm{d} x+\int_{0}^{b} f^{-1}(x) \mathrm{d} x \geq a b \iff b=f(a)
\end{equation}

where $f^{-1}$ is the inverse function of $f$.\\

We shall use Lemma 1 to prove the following Theorem.

Theorem 1 (Holder's inequality). Let $\xi=\left(x_{1}, x_{2}, \ldots, x_{n}\right)$ and $\eta=\left(y_{1}, y_{2}, \ldots, y_{n}\right)$ be arbitrary elements in $\mathbb{R}^{n}$. Then, for $1<p<\infty$,

\begin{equation}\tag{1.2}
\sum_{i=1}^{n}\left|x_{i} y_{i}\right| \leq\left(\sum_{i=1}^{n}\left|x_{i}\right|^{p}\right)^{\frac{1}{p}}\left(\sum_{i=1}^{n}\left|y_{i}\right|^{q}\right)^{\frac{1}{q}}
\end{equation}

where

$$
\frac{1}{p}+\frac{1}{q}=1
$$

Proof of Theorem 1. We shall show firstly that

\begin{equation}\tag{1.3}
a b \leq \frac{a^{p}}{p}+\frac{b^{q}}{q}, \quad \text { where } \frac{1}{p}+\frac{1}{q}=1
\end{equation}

To do this, let $f(x)=x^{p-1}$, for $p \in(1, \infty)$, in (1.1). Then $f^{-1}(x)=x^{\frac{1}{p-1}}$, and

$$
\begin{aligned}
\int_{0}^{a} x^{p-1} \mathrm{~d} x+\int_{0}^{b} x^{\frac{1}{p-1}} \mathrm{~d} x & =\left.\frac{x^{p}}{p}\right|_{0} ^{a}+\left.\frac{(p-1) \cdot x^{\frac{p}{p-1}}}{p}\right|_{0} ^{b} \\
& =\frac{a^{p}}{p}+\frac{(p-1) \cdot b^{\frac{p}{p-1}}}{p}
\end{aligned}
$$

If we let $q=\frac{p}{p-1}$, we have that $\frac{1}{p}+\frac{1}{q}=1$. Therefore,

$$
\int_{0}^{a} x^{p-1} \mathrm{~d} x+\int_{0}^{b} x^{\frac{1}{p-1}} \mathrm{~d} x=\frac{a^{p}}{p}+\frac{b^{q}}{q} \geq a b
$$
and thus (1.3) holds true. To prove (1.2), let $\alpha_{k}$ and $\lambda_{k}$ be two sequences such that

$$
\alpha_{k}:=\frac{x_{k}}{\left(\sum_{i=1}^{n}\left|x_{i}\right|^{p}\right)^{\frac{1}{p}}} \text {, and } \lambda_{k}:=\frac{y_{k}}{\left(\sum_{i=1}^{n}\left|y_{i}\right|^{q}\right)^{\frac{1}{q}}}
$$

It follows from (1.3) that

\begin{equation}\tag{1.4}
\left|\alpha_{k}\right|\left|\lambda_{k}\right| \leq \frac{\left|\alpha_{k}\right|^{p}}{p}+\frac{\left|\lambda_{k}\right|^{q}}{q} \text {, which implies that }\left|\alpha_{k} \lambda_{k}\right| \leq \frac{\left|\alpha_{k}\right|^{p}}{p}+\frac{\left|\lambda_{k}\right|^{q}}{q} \text {. }
\end{equation}

Summing both sides of the inequality (1.4) from $k=1$ to $n$, we have

\begin{equation}\tag{1.5}
\sum_{k=1}^{n}\left|\alpha_{k} \lambda_{k}\right| \leq \sum_{k=1}^{n}\left(\frac{\left|\alpha_{k}\right|^{p}}{p}+\frac{\left|\lambda_{k}\right|^{q}}{q}\right)=\sum_{k=1}^{n} \frac{\left|\alpha_{k}\right|^{p}}{p}+\sum_{k=1}^{n} \frac{\left|\lambda_{k}\right|^{q}}{q}
\end{equation}

Evaluating the first part of the sum in (1.5), we obtain

\begin{equation}\tag{1.6}
\begin{aligned}
\sum_{k=1}^{n} \frac{\left|\alpha_{k}\right|^{p}}{p} & =\sum_{k=1}^{n} \frac{1}{p}\left(\left|\frac{x_{k}}{\left(\sum_{i=1}^{n}\left|x_{i}\right|^{p}\right)^{\frac{1}{p}}}\right|^{p}\right)=\frac{1}{p} \sum_{k=1}^{n} \frac{\left|x_{k}\right|^{p}}{\left(\left.\left|\sum_{i=1}^{n}\right| x_{i}\right|^{p} \mid\right)} \\
& =\frac{1}{p} \frac{\sum_{k=1}^{n}\left|x_{k}\right|^{p}}{\sum_{i=1}^{n}\left|x_{i}\right|^{p}}=\frac{1}{p} .
\end{aligned}
\end{equation}

Similarly, for the second part of the sum in (1.5), we derive

\begin{equation}\tag{1.7}
\begin{aligned}
\sum_{k=1}^{n} \frac{\left|\lambda_{k}\right|^{q}}{q} & =\sum_{k=1}^{n} \frac{1}{q}\left(\left|\frac{y_{k}}{\left(\sum_{i=1}^{n}\left|y_{i}\right|^{q}\right)^{\frac{1}{q}}}\right|^{q}\right)=\frac{1}{q} \sum_{k=1}^{n} \frac{\left|y_{k}\right|^{q}}{\left(\left.\left|\sum_{i=1}^{n}\right| y_{i}\right|^{q} \mid\right)} \\
& =\frac{1}{q} \frac{\sum_{k=1}^{n}\left|y_{k}\right|^{q}}{\sum_{i=1}^{n}\left|y_{i}\right|^{q}}=\frac{1}{q} .
\end{aligned}
\end{equation}

Taking account of (1.6) and (1.7), (1.5) becomes

$$
\sum_{k=1}^{n}\left|\alpha_{k} \lambda_{k}\right| \leq \frac{1}{p}+\frac{1}{q}=1
$$

Therefore,

\begin{equation}\tag{1.8}
\sum_{k=1}^{n}\left|\frac{x_{k}}{\left(\sum_{i=1}^{n}\left|x_{i}\right|^{p}\right)^{\frac{1}{p}}} \cdot \frac{y_{k}}{\left(\sum_{i=1}^{n}\left|y_{i}\right|^{q}\right)^{\frac{1}{q}}}\right|=\frac{\sum_{k=1}^{n}\left|x_{k} y_{k}\right|}{\left(\sum_{i=1}^{n}\left|x_{i}\right|^{p}\right)^{\frac{1}{p}}\left(\sum_{i=1}^{n}\left|y_{i}\right|^{q}\right)^{\frac{1}{q}}} \leq 1
$$

Multiplying both sides of (1.8) by $\left(\sum_{i=1}^{n}\left|x_{i}\right|^{p}\right)^{\frac{1}{p}}\left(\sum_{i=1}^{n}\left|y_{i}\right|^{q}\right)^{\frac{1}{q}}$, we obtain

\begin{equation}\tag{1.9}
\sum_{k=1}^{n}\left|x_{k} y_{k}\right| \leq\left(\sum_{i=1}^{n}\left|x_{i}\right|^{p}\right)^{\frac{1}{p}}\left(\sum_{i=1}^{n}\left|y_{i}\right|^{q}\right)^{\frac{1}{q}}
\end{equation}

Since an index of summation is a dummy variable and is immaterial, we can change the index of summation on the left side of (1.9) from $k$ to $i$ and have that

\begin{equation}\tag{1.10}
\sum_{i=1}^{n}\left|x_{i} y_{i}\right| \leq\left(\sum_{i=1}^{n}\left|x_{i}\right|^{p}\right)^{\frac{1}{p}}\left(\sum_{i=1}^{n}\left|y_{i}\right|^{q}\right)^{\frac{1}{q}}
\end{equation}

Equality holds in (1.10) for $\xi, \eta=\underbrace{(0,0, \cdots, 0)}_{n \text { tuples }}:=\overline{0}$. This concludes the proof of Theorem 1. \newpage
\subsection{Tutorial IV}
\large{}
1. Verify that $
\langle\xi, \eta\rangle=\sum_{i=1}^{n} x_{i} y_{i}$
satisfies the properties of inner product.\\\\
\Large{Solution}\\
We need to show that $\forall_{\xi,\eta,\zeta}\in \mathbb{R}^n$ the following properties are satisfied.
\begin{enumerate}
  \item $\langle\xi, \xi\rangle \geq 0$ and $\langle\xi, \xi\rangle= 0 \iff \xi= \Bar{0}$
  \item $\langle\eta, \xi\rangle=\langle\xi, \eta\rangle$
  \item $\langle\xi+\eta, \zeta\rangle =\langle\xi, \zeta\rangle+\langle\eta, \zeta\rangle$ and $\langle\xi, \eta+\zeta\rangle=\langle\xi, \eta\rangle +\langle\xi,\zeta\rangle$
   \item $\langle\lambda \xi, \eta\rangle=\langle\xi, \lambda \eta\rangle $
  
\end{enumerate}
 \textbf{Proof}
$$Let~~\xi=\left(x_{1}, x_{2}, \ldots, x_{n}\right), \eta=\left(y_{1}, y_{2}, \ldots, y_{n}\right), \zeta=\left(z_{1}, z_{2}, \ldots, z_{n}\right)$$ be arbitrary elements in $\mathbb{R}^{n}$ and $\lambda \in \mathbb{R}$. Then
$$
\langle\xi, \xi\rangle=\sum_{i=1}^{n} x_{i} x_{i}=\sum_{i=1}^{n} x_{i}^{2}
$$

it is obvious $x_{i}^{2} \geq 0$, for all $i=1, 2, 3...n$,\\

 $$Therefore,~~\sum_{i=1}^{n} x_{i}^{2} \geq 0 \implies \langle\xi, \xi\rangle \geq 0$$

$$
\begin{aligned}
\langle\xi, \xi\rangle=0 \iff \sum_{i=1}^{n} x_{i} x_{i}=0 \iff \sum_{i=1}^{n} x_{i}^{2}=0 \iff x_{i}^{2}=0\\
x_{i}^{2}=0~~~ \forall_{i \in \{1,2,3...n\}} \iff x_i=0~~ \forall_{i \in \{1,2,3...n\}} \iff \xi=\Bar{0}\\
So,~ \langle\xi, \xi\rangle=0 \iff \xi=\Bar{0} 
\end{aligned}
$$
\textbf{Hence property(1) hold good}.

$$
\langle\xi, \eta\rangle=\sum_{i=1}^{n} x_{i} y_{i}=\sum_{i=1}^{n} y_{i} x_{i}=\langle\eta, \xi\rangle \implies \langle\xi, \eta\rangle=\langle\eta, \xi\rangle$$
\textbf{Hence property(2) hold good}.\\
\newpage
Considering $\langle\xi, \eta+\zeta\rangle$, we have

$$
\begin{aligned}
\langle\xi, \eta+\zeta\rangle & =\sum_{i=1}^{n} x_{i}\left(y_{i}+z_{i}\right)=\sum_{i=1}^{n} x_{i} y_{i}+\sum_{i=1}^{n} x_{i} z_{i} =\langle\xi, \eta\rangle+\langle\xi, \zeta\rangle .
\end{aligned}
$$
Considering $\langle\xi+\eta, \zeta\rangle$, we have
$$
\begin{aligned}
\langle\xi+\eta, \zeta\rangle & =\sum_{i=1}^{n}\left(x_{i}+y_{i}\right) z_{i}=\sum_{i=1}^{n} x_{i} z_{i}+\sum_{i=1}^{n} y_{i} z_{i}  =\langle\xi, \zeta\rangle+\langle\eta, \zeta\rangle .
\end{aligned}
$$
\textbf{Hence property(3) hold good}.\\

Considering $\langle\lambda \xi, \eta\rangle$, we have

$$
\begin{align}
\langle\lambda \xi, \eta\rangle=\sum_{i=1}^{n} \lambda x_{i} y_{i}=\sum_{i=1}^{n} x_{i}\left(\lambda y_{i}\right)=\langle\xi, \lambda \eta\rangle\\
\implies \langle\lambda \xi, \eta\rangle=\langle\xi, \lambda \eta\rangle 
\end{align}
$$\\

\textbf{Hence, $\langle\xi, \eta\rangle=\sum_{i=1}^{n} x_{i} y_{i}$ satisfies the properties of inner product.}\\

\subsection{Tutorial V}

Using the inequality  $
\langle\xi, \eta\rangle \leq\|\xi\|_{0} \cdot\|\eta\|_{0}
$

\begin{center}
show that   $\|\xi+\eta\|_{0} \leq\|\xi\|_{0}+\|\eta\|_{0}, \forall \xi, \eta \in \mathbb{R}^{n}$
\end{center}

\textbf{SOLUTION}\\
\textbf{Proof.} \\
Let $\xi=\left(x_{1}, x_{2}, \ldots, x_{n}\right)$ and $\eta=\left(y_{1}, y_{2}, \ldots, y_{n}\right)$ be arbitrary elements in $\mathbb{R}^{n}$. Then


$$
\begin{aligned}
\|\xi+\eta\|_{0}^{2} & =\sum_{i=1}^{n}\left(x_{i}+y_{i}\right)^{2}=\sum_{i=1}^{n}\left(x_{i}^{2}+y_{i}^{2}+2 x_{i} y_{i}\right) \\
& =\sum_{i=1}^{n} x_{i}^{2}+\sum_{i=1}^{n} y_{i}^{2}+2 \sum_{i=1}^{n} x_{i} y_{i}=\|\xi\|_{0}^2+\|\eta\|_{0}^2+2\langle\xi, \eta\rangle \\
& \leq\|\xi\|_{0}^2+\|\eta\|_{0}^2+2\|\xi\|_{0}\|\eta\|_{0}\text{(Factorize it)}\\
&
\leq \left(\|\xi\|_{0}+\|\eta\|_{0}\right)^{2}
\end{aligned}
$$

Since $\|\xi\|_{0} \geq 0$, we can take square root and we will have

$$
\|\xi+\eta\|_{0} \leq\|\xi\|_{0}+\|\eta\|_{0}
$$
Thus, completing the proof.
\end{document}
