\documentclass{article}
\usepackage[utf8]{inputenc}
\usepackage{blindtext}
\usepackage[a4paper, total={7in, 10in},left=10mm,]{geometry}
\usepackage{tikz}
\usepackage{bbding}
\usepackage{pifont}
\usepackage{wasysym}
\usepackage{amssymb}
\usepackage{amsmath,amssymb}
\usepackage{mathptmx}
\usetikzlibrary{shapes.geometric}
\usepackage[nomessages]{fp}% http://ctan.org/pkg/fp
\definecolor{B}{HTML}{2E79B2}
\definecolor{W}{HTML}{FF0000}
\usepackage{enumerate}
\usepackage{parcolumns}
\setcounter{section}{-1}

\title{GROUP 1}
\author{Project 1}

\begin{document}
\section{Rotation }


\newcommand{\labelStyle}[4]{
\tikzstyle{every node}=[draw,shape=circle,color=red];
%\node (v0) at (0:0) {$v_0$};
\node[top color=white] (v1) at (0:2.5){$#1$} ;
\node[top color=white] (v2) at (90:2.5) {$#2$};
\node[top color=white] (v3) at (2*90:2.5) {$#3$};
\node[top color=white] (v4) at (3*90:2.5) {$#4$};
}

\newcommand{\Square}{
\begin{tikzpicture}
\draw[step=1cm,gray,very thin] (-4,-4) grid (4,4);
\foreach \x in {-4,-3,-2,-1,0,1,2,3,4}
   \draw (\x cm,1pt) -- (\x cm,-1pt) node[anchor=north] {$\x$};
\foreach \y in {-4,-3,-2,-1,0,1,2,3,4}
    \draw (1pt,\y cm) -- (-1pt,\y cm) node[anchor=east] {$\y$};
    \draw [left color=blue,right color=red](0,2) -- (2,0) -- (0,-2) -- (-2,0) -- (0,2);

\labelStyle{V_2}{V_1}{V_4}{V_3};
\end{tikzpicture}
}

\subsection{}{Rotation at angle $0^0$ anticlockwise}

\Square{}\Square{}

\subsection{}{Rotation at angle $90^0$ anticlockwise}

\Square{}
\begin{tikzpicture}
\draw[step=1cm,gray,very thin] (-4,-4) grid (4,4);
\foreach \x in {-4,-3,-2,-1,0,1,2,3,4}
   \draw (\x cm,1pt) -- (\x cm,-1pt) node[anchor=north] {$\x$};
\foreach \y in {-4,-3,-2,-1,0,1,2,3,4}
    \draw (1pt,\y cm) -- (-1pt,\y cm) node[anchor=east] {$\y$};
    \draw [bottom color=blue,top color=red](0,2) -- (2,0) -- (0,-2) -- (-2,0) -- (0,2);
    \labelStyle{V_3}{V_2}{V_1}{V_4};
\end{tikzpicture}

\subsection{Rotation at angle $180^0$ anticlockwise}
\Square{}
\begin{tikzpicture}
\draw[step=1cm,gray,very thin] (-4,-4) grid (4,4);
\foreach \x in {-4,-3,-2,-1,0,1,2,3,4}
   \draw (\x cm,1pt) -- (\x cm,-1pt) node[anchor=north] {$\x$};
\foreach \y in {-4,-3,-2,-1,0,1,2,3,4}
    \draw (1pt,\y cm) -- (-1pt,\y cm) node[anchor=east] {$\y$};
   \draw [right color=blue,left color=red](0,2) -- (2,0) -- (0,-2) -- (-2,0) -- (0,2);
     \labelStyle{V_4}{V_3}{V_2}{V_1};
\end{tikzpicture}

\subsection{Rotation at angle $270^0$ anticlockwise}
\Square{}
\begin{tikzpicture}

\draw[step=1cm,gray,very thin] (-4,-4) grid (4,4);
\foreach \x in {-4,-3,-2,-1,0,1,2,3,4}
   \draw (\x cm,1pt) -- (\x cm,-1pt) node[anchor=north] {$\x$};
\foreach \y in {-4,-3,-2,-1,0,1,2,3,4}
    \draw (1pt,\y cm) -- (-1pt,\y cm) node[anchor=east] {$\y$};
    
\draw [top color=blue,bottom color=red](0,2) -- (2,0) -- (0,-2) -- (-2,0) -- (0,2) ;
\labelStyle{V_1}{V_4}{V_3}{V_2};
\end{tikzpicture}

\begin{tikzpicture}
\draw[step=1cm,gray,very thin] (-3,-3) grid (3,3);
\foreach \x in {-3,-2,...,3}
   \draw (\x cm,1pt) -- (\x cm,-1pt) node[anchor=north] {$\x$};
\foreach \y in {-3,-2,...,3}
    \draw (1pt,\y cm) -- (-1pt,\y cm) node[anchor=east] {$\y$};
    
\node (p) [draw,rotate=90,minimum size=3cm,regular polygon, regular polygon sides=6] at (0,0) {};

\foreach \n [count=\nu from 1, remember=\n as \lastn, evaluate={\nu+\lastn}] in {1,2,...,6}
\node[anchor=\n*(360/9)]at(p.corner \n){$V_{\nu}$};
\end{tikzpicture}
\newpage









\section{PROJECT 1}

\subsection{Tutorial III}
\large{}
1. Let $\xi \in \mathbb{R}^{n}$ and $\|\cdot\|_{0},\|\cdot\|_{1}$ and $\|\cdot\|_{n}$ be norms on $\mathbb{R}^{n}$ defined by

$$
\|\xi\|_{0}=\left(\sum_{i=1}^{n} x_{i}^{2}\right)^{\frac{1}{2}}, \quad\|\xi\|_{1}=\sum_{i=1}^{n}\left|x_{i}\right|, \quad\|\xi\|_{n}=\max _{1 \leq i \leq n}\left|x_{i}\right|
$$

where $\xi=\left(x_{1}, x_{2}, \ldots, x_{n}\right)$, with $x_{i} \in \mathbb{R}, i=1,2, \ldots, n$.
\begin{itemize}
    \item Prove that
$$
\frac{1}{\sqrt{n}}\|\xi\|_{0} \leq\|\xi\|_{n} \leq\|\xi\|_{1}, \quad \forall_{\xi} \in \mathbb{R}^{n}
$$
\item Hence, deduce that the function

$$
f:\left(\mathbb{R}^{n},\|\cdot\|_{1}\right) \longrightarrow\left(\mathbb{R}^{n},\|\cdot\|_{0}\right)~~~~~~~~is ~continuous~ on ~~\mathbb{R}^{n}
$$
\end{itemize}

\textbf{Solution}\\

let $x_k$ be any entries in $\xi$ such that $1 \leq k \leq n$, it is obvious that

$$
x_{k}^{2} \leq\left(\left|x_{1}\right|+\left|x_{2}\right|+\cdots+\left|x_{n}\right|\right)^{2} .
$$
which can be rewrite in summation notation as :

$$\begin{equation}
x_{k}^{2} \leq \left(\sum_{i=1}^{n}\left|x_{i}\right|\right)^{2}
\end{equation}$$

Now Summing both sides of the equation(1) from $k=1$ to $n$, we have

$$\begin{equation}
\sum_{k=1}^{n} x_{k}^{2} \leq \sum_{k=1}^{n}\left(\sum_{i=1}^{n}\left|x_{i}\right|\right)^{2}
\end{equation}$$

Since $k$ is a dummy variable on the left side of equation(2), we can change it to $i$ and have

$$
\sum_{i=1}^{n} x_{i}^{2} \leq \sum_{k=1}^{n}\left(\sum_{i=1}^{n}\left|x_{i}\right|\right)^{2}
$$

Shifting the first summation to the start we have
$$
\begin{gathered}
\sum_{i=1}^{n} x_{i}^{2} \leq\left(\sum_{i=1}^{n}\left|x_{i}\right|\right)^{2} \sum_{k=1}^{n} 1 \implies
\sum_{i=1}^{n} x_{i}^{2} \leq n\left(\sum_{i=1}^{n}\left|x_{i}\right|\right)^{2}  
\end{gathered}~ ~~\left(since ~\sum_{k=1}^{n} 1=n\right)
$$

Now taking the square root of both side, we have
$$
\begin{aligned}
& \left(\sum_{i=1}^{n} x_{i}^{2}\right)^{\frac{1}{2}} \leq \sqrt{n} \sum_{i=1}^{n}\left|x_{i}\right|
\end{aligned}
$$

Divide both side by $\sqrt{n}$ we have
$$ \frac{1}{\sqrt{n}}\left(\sum_{i=1}^{n} x_{i}^{2}\right)^{\frac{1}{2}} \leq \sum_{i=1}^{n}\left|x_{i}\right|
$$

Recall that, $$\|\xi\|_{0}=\left(\sum_{i=1}^{n} x_{i}^{2}\right)^{\frac{1}{2}} ~and ~\quad\|\xi\|_{1}=\sum_{i=1}^{n}\left|x_{i}\right|$$

Therefore,

$$\frac{1}{\sqrt{n}}\left(\sum_{i=1}^{n} x_{i}^{2}\right)^{\frac{1}{2}} \leq \sum_{i=1}^{n}\left|x_{i}\right| .
\implies
\frac{1}{\sqrt{n}}\|\xi\|_{0} \leq\|\xi\|_{1}
$$\\

Firsty let us  show that

$$
\|\xi\|_{n} \leq\|\xi\|_{1}
$$

It is Obvious that

$$
\left|x_{1}\right|+\left|x_{2}\right|+\cdots+\left|x_{n}\right| \geq \max _{1 \leq i \leq n}\left|x_{i}\right| .
$$

the above equation can be rewrite in summation form as:

$$
\sum_{i=1}^{n}\left|x_{i}\right| \geq \max _{1 \leq i \leq n}\left|x_{i}\right|
$$

this shows that
$$
\|\xi\|_{n} \leq\|\xi\|_{1} \text {. }
$$

Now we need to show that.
$$
\frac{1}{\sqrt{n}}\|\xi\|_{0} \leq\|\xi\|_{n}
$$
we know that

$$
x_{i}^{2} \leq \max _{1 \leq i \leq n}\left|x_{i}\right|^{2}
$$

$$ \text{But} \max _{1 \leq i \leq n}\left|x_{i}\right|^{2}=\left(\max _{1 \leq i \leq n}\left|x_{i}\right|\right)^{2}$$
So, $$
x_{i}^{2} \leq \max _{1 \leq i \leq n}\left|x_{i}\right|^{2} \iff x_{i}^{2} \leq\left(\max _{1 \leq i \leq n}\left|x_{i}\right|\right)^{2}
$$

Take summation of both sides of from $i=1$ to $n$, we have

$$
\sum_{i=1}^{n} x_{i}^{2} \leq \sum_{i=1}^{n}\left(\max _{1 \leq i \leq n}\left|x_{i}\right|\right)^{2}
$$

Since the maximum function on the right side  is not depending on $i$, then we can shift it.

$$
\sum_{i=1}^{n} x_{i}^{2} \leq\left(\max _{1 \leq i \leq n}\left|x_{i}\right|\right)^{2} \sum_{i=1}^{n} 1
$$
$$\text{Since } ~~\sum_{i=1}^{n} 1 =n ~~\text{ then we have,} $$ 

$$
\sum_{i=1}^{n} x_{i}^{2} \leq n\left(\max _{1 \leq i \leq n}\left|x_{i}\right|\right)^{2}
$$

Since both sides are positive, we can take square root and obtain the following:

$$
\left(\sum_{i=1}^{n} x_{i}^{2}\right)^{\frac{1}{2}} \leq \sqrt{n} \max _{1 \leq i \leq n}\left|x_{i}\right|, $$
divide both side by $\sqrt{n}$ we have:
$$\quad \frac{1}{\sqrt{n}}\left(\sum_{i=1}^{n} x_{i}^{2}\right)^{\frac{1}{2}} \leq \max _{1 \leq i \leq n}\left|x_{i}\right|
$$

Showing that

$$
\frac{1}{\sqrt{n}}\|\xi\|_{0} \leq\|\xi\|_{n}
$$

The inequalities

$$
\|\xi\|_{n} \leq\|\xi\|_{1}, \quad \frac{1}{\sqrt{n}}\|\xi\|_{0} \leq\|\xi\|_{n}, \quad \frac{1}{\sqrt{n}}\|\xi\|_{0} \leq\|\xi\|_{1}
$$

implies that
$$\frac{1}{\sqrt{n}}\|\xi\|_{0} \leq\|\xi\|_{n} \leq\|\xi\|_{1}, \quad \forall \xi \in \mathbb{R}^{n}$$

\section{Tutorial Solution}
\section{Proof of Holder's inequality}
\large
\textbf{Lemma 1:} (Young's inequality). Let $f$ be a real-valued, continuous, and strictly increasing function on $[0, c]$ with $c>0$. If $f(0)=0, a \in[0, c]$, and $b \in[0, f(c)]$, then

$$
\int_{0}^{a} f(x) \mathrm{d} x+\int_{0}^{b} f^{-1}(x) \mathrm{d} x \geq a b \iff b=f(a)
$$
where $f^{-1}$ is the inverse function of $f$.\\

We shall use Lemma 1 to prove the following Theorem.

Theorem 1 (Hölder's inequality). Let $\xi=\left(x_{1}, x_{2}, \ldots, x_{n}\right)$ and $\eta=\left(y_{1}, y_{2}, \ldots, y_{n}\right)$ be arbitrary elements in $\mathbb{R}^{n}$. Then, for $1<p<\infty$,

$$
\sum_{i=1}^{n}\left|x_{i} y_{i}\right| \leq\left(\sum_{i=1}^{n}\left|x_{i}\right|^{p}\right)^{\frac{1}{p}}\left(\sum_{i=1}^{n}\left|y_{i}\right|^{q}\right)^{\frac{1}{q}}
$$

where

$$
\frac{1}{p}+\frac{1}{q}=1
$$

Proof of Theorem 1. We shall show firstly that

$$
a b \leq \frac{a^{p}}{p}+\frac{b^{q}}{q}, \quad \text { where } \frac{1}{p}+\frac{1}{q}=1
$$

To do this, let $f(x)=x^{p-1}$, for $p \in(1, \infty)$, in (1.1). Then $f^{-1}(x)=x^{\frac{1}{p-1}}$, and

$$
\begin{aligned}
\int_{0}^{a} x^{p-1} \mathrm{~d} x+\int_{0}^{b} x^{\frac{1}{p-1}} \mathrm{~d} x & =\left.\frac{x^{p}}{p}\right|_{0} ^{a}+\left.\frac{(p-1) \cdot x^{\frac{p}{p-1}}}{p}\right|_{0} ^{b} \\
& =\frac{a^{p}}{p}+\frac{(p-1) \cdot b^{\frac{p}{p-1}}}{p}
\end{aligned}
$$

If we let $q=\frac{p}{p-1}$, we have that $\frac{1}{p}+\frac{1}{q}=1$. Therefore,

$$
\int_{0}^{a} x^{p-1} \mathrm{~d} x+\int_{0}^{b} x^{\frac{1}{p-1}} \mathrm{~d} x=\frac{a^{p}}{p}+\frac{b^{q}}{q} \geq a b
$$
and thus (1.3) holds true. To prove (1.2), let $\alpha_{k}$ and $\lambda_{k}$ be two sequences such that

$$
\alpha_{k}:=\frac{x_{k}}{\left(\sum_{i=1}^{n}\left|x_{i}\right|^{p}\right)^{\frac{1}{p}}} \text {, and } \lambda_{k}:=\frac{y_{k}}{\left(\sum_{i=1}^{n}\left|y_{i}\right|^{q}\right)^{\frac{1}{q}}}
$$

It follows from (1.3) that

$$
\left|\alpha_{k}\right|\left|\lambda_{k}\right| \leq \frac{\left|\alpha_{k}\right|^{p}}{p}+\frac{\left|\lambda_{k}\right|^{q}}{q} \text {, which implies that }\left|\alpha_{k} \lambda_{k}\right| \leq \frac{\left|\alpha_{k}\right|^{p}}{p}+\frac{\left|\lambda_{k}\right|^{q}}{q} \text {. }
$$

Summing both sides of the inequality (1.4) from $k=1$ to $n$, we have

$$
\sum_{k=1}^{n}\left|\alpha_{k} \lambda_{k}\right| \leq \sum_{k=1}^{n}\left(\frac{\left|\alpha_{k}\right|^{p}}{p}+\frac{\left|\lambda_{k}\right|^{q}}{q}\right)=\sum_{k=1}^{n} \frac{\left|\alpha_{k}\right|^{p}}{p}+\sum_{k=1}^{n} \frac{\left|\lambda_{k}\right|^{q}}{q}
$$

Evaluating the first part of the sum in (1.5), we obtain

$$
\begin{aligned}
\sum_{k=1}^{n} \frac{\left|\alpha_{k}\right|^{p}}{p} & =\sum_{k=1}^{n} \frac{1}{p}\left(\left|\frac{x_{k}}{\left(\sum_{i=1}^{n}\left|x_{i}\right|^{p}\right)^{\frac{1}{p}}}\right|^{p}\right)=\frac{1}{p} \sum_{k=1}^{n} \frac{\left|x_{k}\right|^{p}}{\left(\left.\left|\sum_{i=1}^{n}\right| x_{i}\right|^{p} \mid\right)} \\
& =\frac{1}{p} \frac{\sum_{k=1}^{n}\left|x_{k}\right|^{p}}{\sum_{i=1}^{n}\left|x_{i}\right|^{p}}=\frac{1}{p} .
\end{aligned}
$$

Similarly, for the second part of the sum in (1.5), we derive

$$
\begin{aligned}
\sum_{k=1}^{n} \frac{\left|\lambda_{k}\right|^{q}}{q} & =\sum_{k=1}^{n} \frac{1}{q}\left(\left|\frac{y_{k}}{\left(\sum_{i=1}^{n}\left|y_{i}\right|^{q}\right)^{\frac{1}{q}}}\right|^{q}\right)=\frac{1}{q} \sum_{k=1}^{n} \frac{\left|y_{k}\right|^{q}}{\left(\left.\left|\sum_{i=1}^{n}\right| y_{i}\right|^{q} \mid\right)} \\
& =\frac{1}{q} \frac{\sum_{k=1}^{n}\left|y_{k}\right|^{q}}{\sum_{i=1}^{n}\left|y_{i}\right|^{q}}=\frac{1}{q} .
\end{aligned}
$$

Taking account of (1.6) and (1.7), (1.5) becomes

$$
\sum_{k=1}^{n}\left|\alpha_{k} \lambda_{k}\right| \leq \frac{1}{p}+\frac{1}{q}=1
$$

Therefore,

$$
\sum_{k=1}^{n}\left|\frac{x_{k}}{\left(\sum_{i=1}^{n}\left|x_{i}\right|^{p}\right)^{\frac{1}{p}}} \cdot \frac{y_{k}}{\left(\sum_{i=1}^{n}\left|y_{i}\right|^{q}\right)^{\frac{1}{q}}}\right|=\frac{\sum_{k=1}^{n}\left|x_{k} y_{k}\right|}{\left(\sum_{i=1}^{n}\left|x_{i}\right|^{p}\right)^{\frac{1}{p}}\left(\sum_{i=1}^{n}\left|y_{i}\right|^{q}\right)^{\frac{1}{q}}} \leq 1
$$

Multiplying both sides of (1.8) by $\left(\sum_{i=1}^{n}\left|x_{i}\right|^{p}\right)^{\frac{1}{p}}\left(\sum_{i=1}^{n}\left|y_{i}\right|^{q}\right)^{\frac{1}{q}}$, we obtain

$$
\sum_{k=1}^{n}\left|x_{k} y_{k}\right| \leq\left(\sum_{i=1}^{n}\left|x_{i}\right|^{p}\right)^{\frac{1}{p}}\left(\sum_{i=1}^{n}\left|y_{i}\right|^{q}\right)^{\frac{1}{q}}
$$

Since an index of summation is a dummy variable and is immaterial, we can change the index of summation on the left side of (1.9) from $k$ to $i$ and have that

$$
\sum_{i=1}^{n}\left|x_{i} y_{i}\right| \leq\left(\sum_{i=1}^{n}\left|x_{i}\right|^{p}\right)^{\frac{1}{p}}\left(\sum_{i=1}^{n}\left|y_{i}\right|^{q}\right)^{\frac{1}{q}}
$$

Equality holds in (1.10) for $\xi, \eta=\underbrace{(0,0, \cdots, 0)}_{n \text { tuples }}:=\overline{0}$. This concludes the proof of Theorem 1.


\newpage
\subsection{Tutorial II}
\large{}
1. Verify that $
\langle\xi, \eta\rangle=\sum_{i=1}^{n} x_{i} y_{i}$
satisfies the properties of inner product.\\\\
\Large{Solution}\\
We need to show that $\forall_{\xi,\eta,\zeta}\in \mathbb{R}^n$ the following properties are satisfied.
\begin{enumerate}
  \item $\langle\xi, \xi\rangle \geq 0$ and $\langle\xi, \xi\rangle= 0 \iff \xi= \Bar{0}$
  \item $\langle\eta, \xi\rangle=\langle\xi, \eta\rangle$
  \item $\langle\xi+\eta, \zeta\rangle =\langle\xi, \zeta\rangle+\langle\eta, \zeta\rangle$ and $\langle\xi, \eta+\zeta\rangle=\langle\xi, \eta\rangle +\langle\xi,\zeta\rangle$
   \item $\langle\lambda \xi, \eta\rangle=\langle\xi, \lambda \eta\rangle $
  
\end{enumerate}
 \textbf{Proof}\\

$$Let~~\xi=\left(x_{1}, x_{2}, \ldots, x_{n}\right), \eta=\left(y_{1}, y_{2}, \ldots, y_{n}\right), \zeta=\left(z_{1}, z_{2}, \ldots, z_{n}\right)$$ be arbitrary elements in $\mathbb{R}^{n}$ and $\lambda \in \mathbb{R}$. Then
$$
\langle\xi, \xi\rangle=\sum_{i=1}^{n} x_{i} x_{i}=\sum_{i=1}^{n} x_{i}^{2}
$$

it is obvious $x_{i}^{2} \geq 0$, for all $i=1, 2, 3...n$,\\

 $$Therefore,~~\sum_{i=1}^{n} x_{i}^{2} \geq 0 \implies \langle\xi, \xi\rangle \geq 0$$

$$
\begin{aligned}
\langle\xi, \xi\rangle=0 \iff \sum_{i=1}^{n} x_{i} x_{i}=0 \iff \sum_{i=1}^{n} x_{i}^{2}=0 \iff x_{i}^{2}=0\\
x_{i}^{2}=0~~~ \forall_{i \in \{1,2,3...n\}} \iff x_i=0~~ \forall_{i \in \{1,2,3...n\}} \iff \xi=\Bar{0}\\
So,~ \langle\xi, \xi\rangle=0 \iff \xi=\Bar{0} 
\end{aligned}
$$
\textbf{Hence property(1) hold good}.

$$
\langle\xi, \eta\rangle=\sum_{i=1}^{n} x_{i} y_{i}=\sum_{i=1}^{n} y_{i} x_{i}=\langle\eta, \xi\rangle \implies \langle\xi, \eta\rangle=\langle\eta, \xi\rangle$$
\textbf{Hence property(2) hold good}.\\
\newpage
Considering $\langle\xi, \eta+\zeta\rangle$, we have

$$
\begin{aligned}
\langle\xi, \eta+\zeta\rangle & =\sum_{i=1}^{n} x_{i}\left(y_{i}+z_{i}\right)=\sum_{i=1}^{n} x_{i} y_{i}+\sum_{i=1}^{n} x_{i} z_{i} =\langle\xi, \eta\rangle+\langle\xi, \zeta\rangle .
\end{aligned}
$$
Considering $\langle\xi+\eta, \zeta\rangle$, we have
$$
\begin{aligned}
\langle\xi+\eta, \zeta\rangle & =\sum_{i=1}^{n}\left(x_{i}+y_{i}\right) z_{i}=\sum_{i=1}^{n} x_{i} z_{i}+\sum_{i=1}^{n} y_{i} z_{i}  =\langle\xi, \zeta\rangle+\langle\eta, \zeta\rangle .
\end{aligned}
$$
\textbf{Hence property(3) hold good}.\\

Considering $\langle\lambda \xi, \eta\rangle$, we have

$$
\begin{align}
\langle\lambda \xi, \eta\rangle=\sum_{i=1}^{n} \lambda x_{i} y_{i}=\sum_{i=1}^{n} x_{i}\left(\lambda y_{i}\right)=\langle\xi, \lambda \eta\rangle\\
\implies \langle\lambda \xi, \eta\rangle=\langle\xi, \lambda \eta\rangle ~(Property ~ 4~satisfied)
\end{align}
$$\\

\textbf{Hence, $\langle\xi, \eta\rangle=\sum_{i=1}^{n} x_{i} y_{i}$ satisfies the properties of inner product.}\\

\subsection{Tutorial II}

Using the inequality

$$
\langle\xi, \eta\rangle \leq\|\xi\|_{0} \cdot\|\eta\|_{0}
$$

show that

$$
\|\xi+\eta\|_{0} \leq\|\xi\|_{0}+\|\eta\|_{0}, \forall \xi, \eta \in \mathbb{R}^{n}
$$

Proof. Let $\xi=\left(x_{1}, x_{2}, \ldots, x_{n}\right)$ and $\eta=\left(y_{1}, y_{2}, \ldots, y_{n}\right)$ be arbitrary elements in $\mathbb{R}^{n}$. Then

$$
\begin{aligned}
\|\xi+\eta\|_{0}^{2} & =\sum_{i=1}^{n}\left(x_{i}+y_{i}\right)^{2}=\sum_{i=1}^{n}\left(x_{i}^{2}+y_{i}^{2}+2 x_{i} y_{i}\right) \\
& =\sum_{i=1}^{n} x_{i}^{2}+\sum_{i=1}^{n} y_{i}^{2}+2 \sum_{i=1}^{n} x_{i} y_{i}=\|\xi\|_{0}+\|\eta\|_{0}+2\langle\xi, \eta\rangle \\
& \leq\|\xi\|_{0}+\|\eta\|_{0}+2\|\xi\|_{0}\|\eta\|_{0}=\left(\|\xi\|_{0}+\|\eta\|_{0}\right)^{2}
\end{aligned}
$$

Since $\|\xi\|_{0} \geq 0$, we can take square root and obtain

$$
\|\xi+\eta\|_{0} \leq\|\xi\|_{0}+\|\eta\|_{0}
$$

Thus, completing the proof.


2. If $\xi=\eta$, then we have

$$
\|\xi\|_{0}=\langle\xi, \xi\rangle^{\frac{1}{2}}=\left(\sum_{i=1}^{n} x_{i}^{2}\right)^{\frac{1}{2}}=\|\xi\|_{2}
$$

Proof. Replacing $\eta$ with $\xi$, we derive

$$
\langle\xi, \xi\rangle=\sum_{i=1}^{n} x_{i} \cdot x_{i}=\sum_{i=1}^{n} x_{i}^{2}
$$

Comparing with the Euclidean norm, we have

$$
\|\xi\|_{0}=\langle\xi, \xi\rangle^{\frac{1}{2}}=\left(\sum_{i=1}^{n} x_{i}^{2}\right)^{\frac{1}{2}}=\|\xi\|_{2}
$$

Thus, completing the proof.

\subsection{Tutorial II}

Using the inequality

$$
\langle\xi, \eta\rangle \leq\|\xi\|_{0} \cdot\|\eta\|_{0}
$$

show that

$$
\|\xi+\eta\|_{0} \leq\|\xi\|_{0}+\|\eta\|_{0}, \forall \xi, \eta \in \mathbb{R}^{n}
$$

Proof. Let $\xi=\left(x_{1}, x_{2}, \ldots, x_{n}\right)$ and $\eta=\left(y_{1}, y_{2}, \ldots, y_{n}\right)$ be arbitrary elements in $\mathbb{R}^{n}$. Then

$$
\begin{aligned}
\|\xi+\eta\|_{0}^{2} & =\sum_{i=1}^{n}\left(x_{i}+y_{i}\right)^{2}=\sum_{i=1}^{n}\left(x_{i}^{2}+y_{i}^{2}+2 x_{i} y_{i}\right) \\
& =\sum_{i=1}^{n} x_{i}^{2}+\sum_{i=1}^{n} y_{i}^{2}+2 \sum_{i=1}^{n} x_{i} y_{i}=\|\xi\|_{0}+\|\eta\|_{0}+2\langle\xi, \eta\rangle \\
& \leq\|\xi\|_{0}+\|\eta\|_{0}+2\|\xi\|_{0}\|\eta\|_{0}=\left(\|\xi\|_{0}+\|\eta\|_{0}\right)^{2}
\end{aligned}
$$

Since $\|\xi\|_{0} \geq 0$, we can take square root and obtain

$$
\|\xi+\eta\|_{0} \leq\|\xi\|_{0}+\|\eta\|_{0}
$$

Thus, completing the proof.


\end{document}
